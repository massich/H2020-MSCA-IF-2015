\section{EXCELLENCE}
\label{sec:excellence}


\subsection{Quality, innovative aspects and credibility of the research}
\label{sec:quality}

\paragraph{Introduction to breast cancer and multiparametric \mri}
Breast cancer has significant impact on the well-being of the female population
worldwide. This disease represents the leading cause of cancer death among
females worldwide\cite{cancerStatistics2011}.  This has motivated the
establishment of \acp{bsp} to facilitate breast cancer detection at an early
stage. Despite X-ray \dm is considered the gold standard technique for \bsp,
other screening techniques like \us and \mri are being investigated to overcome
\dm limitations due to tissue superposition which can either mimic or obscure
malignant pathology, and avoid X-ray radiation all together.

Though \mri is a promising alternative to \dm, \nmle lesions exhibit an
heterogeneous appearance in breast \mri with high variations in kinetic
characteristics and typical morphological parameters
\cite{rosen2007birads,skamato2008categorization,yabuuchi2010nmle}, and have
a lower reported specificity (69$\%$) and sensitivity (75$\%$) than
mass-enhancing lesions \cite{Vag} in \cemri. The diagnosis of \nmle lesions is thus far more challenging.

Malignant lesions such as \dcis and \ilc commonly exhibit a segmental or
linear enhancement pattern and benign lesions such as fibrocystic
changes present as well a \nmle \cite{Vag}.
However, a systematic classification of \nmle
lesions is not in place. A classification of such lesions would be
highly beneficial since they may reduce the biopsies' numbers.
   Recently, there
have been new research initiatives to assess \nmle lesions using
multiparametric \mri which combines
T1-weighted contrast-enhanced \mri with diffusion-weighted imaging
(\dwi) \cite{Pinker1,yabuuchi2010nmle}. It was shown that the combination of
morphological, functional and molecular information offered by
multiparametric \mri improves the diagnostic accuracy  of breast
cancer diagnosis. Another study showed that T2-weighted imaging can
better represent the morphological features of small lesions \cite{Wu1}
and combined with \dwi it increased the diagnostic performance of \mri.

 Computer-aided analysis showed a much lower
sensitivity ( 0.79 vs. 0.97) and specificity (0.56 vs. 0.8) for \nmle lesions
compared with masses and suggested the need for more advanced
algorithms for the diagnosis of \nmle \cite{Newell,Vag,Jansen1,Jansen2}.
Uniformity, and a clear set of imaging descriptors for the reporting of T2 and \dwi features of \nmle is lacking.
Furthermore, there is no quantitative technique  for how to combine the morphological, functional and
molecular information derived from multiparametric imaging.

{\it The bottleneck that remains for providing an improved
differential diagnosis of \nmle lesions and thus
contribute to advancing \cad systems beyond the current  level are
determining
descriptors that incorporate  the diagnostic information from
multiparametric \mri.}
 {\bf Our proposal to develop advanced image analysis algorithms to improve the differential diagnosis
 of the challenging \nmle lesions
would provide the radiologist with a fast and accurate computational diagnosis support.}

\subsection{Clarity and quality of transfer of knowledge/training for the development of the researcher in light of the research objectives}
\label{sec:transfer}


\subsection{Quality of the supervision and the hosting arrangements}
\label{sec:supervision}

Required sub-heading:
\subsubsection*{Qualifications and experience of the supervisor(s)}


\paragraph{Career development}
Employers and/or funders of researchers should draw up, preferably within the framework of their human resources management, a specific career development strategy for researchers at all stages of their career, regardless of their contractual situation, including for researchers on fixed-term contracts. It should include the availability of mentors involved in providing support and guidance for the personal and professional development of researchers, thus motivating them and contributing to reducing any insecurity in their professional future. All researchers should be made familiar with such provisions and arrangements.

\subsection{Capacity of the researcher to reach and re-enforce a position of professional maturity in research}
\label{sec:maturity}

