\section{EXCELLENCE}
\label{sec:excellence}

\subsection{Quality, innovative aspects and credibility of the research}
\label{sec:quality}
\paragraph{Introduction to breast cancer and multiparametric \acs{mri}}

Breast cancer is the leading cause of cancer deaths among females
worldwide~\cite{cancerStatistics2011}. This has motivated the establishment of
\acp{bsp} to facilitate breast cancer detection at an early stage. Despite
X-ray \dm is considered the gold standard technique for \bsp, other screening
techniques like \us and \mri are being investigated to overcome \dm limitations
due to tissue superposition which can either mimic or obscure malignant
pathology, and avoid X-ray radiation all together.

Though \mri is a promising alternative to \dm, \nmle lesions exhibit
a heterogeneous appearance in breast \mri with high variations in kinetic
characteristics and typical morphological parameters
\cite{rosen2007birads,skamato2008categorization,yabuuchi2010nmle}, and have
, and resulting in a lower reported specificity (69\%) and sensitivity
(75$\%$) than mass-enhancing lesions\cite{Vag} in \cemri.  The diagnosis of
\nmle lesions is thus far more challenging.

Malignant lesions such as \dci and \ilc commonly exhibit a segmental or linear
enhancement pattern and benign lesions such as fibrocystic changes present as
well a \nmle \cite{Vag}.  However, a systematic classification of \nmle
lesions is not in place. A classification of such lesions would be highly
beneficial since they may reduce the biopsies’ numbers.  Recently, there have
been new research initiatives to assess \nmle lesions using multiparametric
\mri which combines T1-weighted contrast-enhanced \mri with \dwi \todo{PBH
+ 13, YMK + 10}.

It was shown that the combination of morphological, functional and molecular
information offered by multiparametric \mri improves the diagnostic accuracy of
breast cancer diagnosis.  Another study showed that T2-weighted imaging can
better represent the morphological features of small lesions \todo{WCH + 14}
and combined with \dwi it increased the diagnostic performance of MRI.  \cad
systems showed a much lower sensitivity ( 0.79 vs.
0.97) and specificity (0.56 vs. 0.8) for \nmle lesions compared with masses and
suggested the need for more advanced algorithms for the diagnosis of \nmle
\cite{Newell,Vag,Jansen1,Jansen2}.  Uniformity, and a clear set of
imaging descriptors for the reporting of T2 and \dwi features of \nmle is
lacking. Furthermore, there is no quantitative technique for how to combine the
morphological, functional and molecular information derived from
multiparametric imaging.


% Combinations of morphological and temporal \acs{birads} descriptors have proven
% to be insufficient to aid in the automated differential diagnosis of these
% lesions in \cemri.  Newest clinical studies suggest that T2-weighted image
% sequences and \dwi may provide additional specificity.

% the aim of this fellowship is to translate these findings into a new \cad
% system.
% %we propose to develop novel descriptors derived from the multispectral \mri
% %data that will lead to a substantial improvement in diagnostic accuracy and
% %efficiency and validate them in four specific experiments.
% our hypothesis is that a combination of novel descriptors extracted from
% multiparametric breast \mri has the potential to substantially improve the
% diagnostic value of the detection and classification of \acp{nmle}.

% this first and novel \cad system in multiparametric breast \mri will reduce
% false positive recalls and thus increase specificity.  a reduction in recall of
% only 5\% would already be clinically relevant, considering the costs and
% patient discomfort associated with second look ultrasound examinations and
% biopsies.

%  the experience of \acs{vicorob} in breast-\cad, the preliminary studies in
%  multispectral \mri carried out in \acs{florida}
%  and the clinical support from \acs{udiat} guarantee
%  the success of this project as well as the correct transfer of knowledge from
%  the laboratory to the clinical site.  it is also planned to commercialise the
%  output \cad tool to clinical sites through existing medical imaging companies
%  or via a spin-off.


\emph{The bottleneck that remains for providing an improved differential
diagnosis of \nmle lesions and thus contribute to advancing
\cad systems beyond the current level are determining descriptors that
incorporate the diagnostic information from multiparametric \mri.}
\textbf{Our proposal to develop advanced image analysis algorithms to improve
the differential diagnosis of the challenging \nmle lesions
would provide the radiologist with a fast and accurate computational diagnosis
support.}

\todo[inline]{Here is missing a to state somewhere:\\ best career possibilities for the experiecned researcher and new collaboration opportunities for the host organization(s)}

\paragraph{Research objectives}

With the aid of this fellowship, the experience in Breast-\cad of \vicorob, the preliminary results in \cemri from \florida, and the clinical support from \udiat, we aim to encode multiparametric \mri clinical findings into a new \cad with higher specificity that will reduce the costs and patient discomfort associated with second look examinations and biopsies.
In order to successfully achieve this purpose, the following objectives will be pursued:

\begin{description}
\item [Aim\,1: Develop a novel image regularization framework for \nmle lesions from multiparametric \mri.] \hfill \\
  \todo{maybe make reference to ASURE project}
  The regularization step represents a crucial step for the subsequent feature
  extraction and classification since the images stem from heterogeneous
  sources. A standard preprocessing step is followed by a novel joint
  segmentation and registration algorithm. We propose a novel joint
  segmentation and registration algorithm based on a variational model and
  level set approach which incorporates spatial as well as temporal
  contrast-enhanced images.\todo{maybe MRF, to link with my thesis}
  The multiparametric images are registered such that all segmented images will
  be in the same reference frame.

  \item [Aim\,2: Identifying novel descriptors such as structure tensors and texture from T2-\mri and \ivim maps from \dwi as additional informative discriminators of \nmle lesions] \hfill \\
    The \birads-based\todo{describe BIRADS somewhere?} features from \cemri

    proved to be insufficient to differentiate between malignant and benign for
    \nmle lesions and therefore additional descriptors from multiparametric
    images are needed \todo{PBH + 13}. Furthermore, the lesion heterogeneity is
    insufficiently described by a single ADC\todo{ADC, stands for?} threshold
    and thus more detailed structural and functional image features have to be
    extracted from T2-\mri and \dwi. The proposed novel descriptors include the
    additional information from multiparametric \mri and thus capture the
    structure of the breast tissue in a unique manner like no other method
    before.

  \item [Aim\,3: Identifying novel spatio-temporal descriptors from \cemri as the most powerful discriminators of \nmle lesions]
  \hfill \\

  In the case of \nmle lesions, there is a high variance in morphological and
  kinetic characteristics and as a consequence a high proportion of
  false-positive diagnosis \todo{VBD + 13}. The automated extracted features
  that have been applied to lesion characterization capture either the
  variations in their temporal enhancement or in spatial (morphological)
  structures or they are global features unable to describe local information.
  To address this shortcoming we propose novel mathematical spatiotemporal
  feature descriptors such as local velocity moments, scaling index and dynamic
  texture derived from geometrical multiscale decomposition that are able to
  capture the segmental, focal, linear, regional, and diffuse, and internal
  enhancement patterns (homogeneous, heterogeneous, clumped, clustered ring
  enhancement, dendritic), and lesion heterogeneity and will compare their
  performance together with the features from Aim 2 regarding lesion
  classification.

  \item [Aim\,4: Validation of the proposed system in terms of performance and direct comparison
to that of the radiologists in detecting lesions]
  \hfill \\

  Statistical methods will evaluate its performance as a stand-alone system and
  in comparison with the radiologist’s competence. Adding novel algorithms to
  existing techniques will create a flexible toolbox that can be applied with
  minimal modifications to identifying other type of lesions or monitor
  response to chemotherapy.

  % \item [Aim\,5: ]
  % \hfill \\
  % \emph{\textbf{Impact:} }
\end{description}

\paragraph{Overview of the action}

The proposed 24 month fellowship will develop advanced image analysis algorithms to address the challenge of properly diagnose \nmle lesions in \mri.
Multiparametric \mri information will be inserted for the first time in a new \cad system using previous experience of \vicorob. Consequently the lesion detection in \cad systems will be improved, the false positive recalls reduced and thus a direct impact into society.

To ensure a sufficient volume of data to develop the \cad system, a database available at \florida with more than 400 patient cases of \mri-detected \nmle lesions will be used to start the project. While, the correct performance of the \cad, clinical validation and implantation, will be achieved with the support from expert radiologists from \udiat, where 3 months secondments are planned.

The new research knowledge will be disseminated through open access journals. Finally, this feasible \cad system with potential to improve breast cancer detection will be promoted in technical exhibitions, such as \ecr, in order to search for medical companies interested in distributing such tool in clinical sites. Alternative, the tool can be commercialised via a new spin-off created at \vicorob, where three spin-off companies have already been created.

\paragraph{Research methodology and approach}
\paragraph{Originality and innovative aspects of the research programme}

The proposed research aims at developing an innovative and robust \cad system for the evaluation of \nmle lesions, based on multiparametric \mri; thus increasing specificity without compromising the sensitivity of \cemri.
Our believe is that conventional \mri acquisition protocols are unable to capture the physical properties of \nmle lesions.
Therefore, rather than trying to implement a new \cad methodology to work in regular \mri as other researcher has shown,
our proposal tries to stablish new \mri acquisitions protocol to build multiparametric \mri and improve the breast cancer detection through finding new bio-markers that are not identifiable when using conventional imaging and implementing them as a new \cad system.

Furthermore, the proposed \cad system not only will be developed in a research laboratory as observed in the literature, but it will go beyond and will be tested at the clinical facilities of \udiat to create a commercialised product.

\subsection{Clarity and quality of transfer of knowledge/training for the development of the researcher in light of the research objectives}
\label{sec:transfer}

\paragraph{How the Experienced Researcher will gain new knowledge}

The \emph{Experienced Researcher, Dr.\,Joan Massich,} will be supervised principally by Dr.\,Anke Meyer-Baese and the \emph{Dr.\,Massich}'s former PhD advisor, Dr.\,Joan Mart\'i.
However, they will have the support of the members in both teams: the Dept. of Scientific Computing at \florida, and \vicorob institute at \udg through regular meetings. This will open the possibility to open new collaboration between these teams in other medical fields like brain \mri or prostate cancer, which are medical areas currently being investigated by the two institutions with no collaboration.

Also, \emph{Dr.\,Massich} will be involved in the supervision of research projects of the \emph{Erasmus\,+\,} Master in \maia \todo{http://maia-jointmaster.weebly.com/} and new PhD student. \emph{Dr.\,Massich}, with the supervision and guidance of Dr.\,Meyer-Baese and Dr.\,Mart\'i will hone his research writing skills via writing grant proposals, which will also give a continuity to his research career.



\subsection{Quality of the supervision and the hosting arrangements}
\label{sec:supervision}

\todo[inline]{to modify}
Required sub-heading:
\subsubsection*{Qualifications and experience of the supervisor(s)}

Information regarding the supervisor(s) must include the level of experience on the research topic proposed and document its track record of work, including the main international collaborations. Information provided should include participation in projects, publications, patents and any other relevant results.
To avoid duplication, the role and profile of the supervisor(s) should only be listed in the "Capacity of the Participating Organisations" tables (see section 6 below).
The text must show that the Experienced Researcher should be well integrated within the hosting organisation(s) in order that all parties gain the maximum knowledge and skills from the fellowship.
The following section of the European Charter for Researchers refers specifically to career development:

\paragraph{Career development}
Employers and/or funders of researchers should draw up, preferably within the framework of their human resources management, a specific career development strategy for researchers at all stages of their career, regardless of their contractual situation, including for researchers on fixed-term contracts. It should include the availability of mentors involved in providing support and guidance for the personal and professional development of researchers, thus motivating them and contributing to reducing any insecurity in their professional future. All researchers should be made familiar with such provisions and arrangements.

\subsection{Capacity of the researcher to reach and re-enforce a position of professional maturity in research}
\label{sec:maturity}

Please keep in mind that the fellowships will be awarded to the most talented researchers as shown by their ideas and their track record, where it is a fair indicator given their level of experience.
\subsection{Quality of the supervision and the hosting arrangements}
\label{sec:supervision}

\todo[inline]{to modify}
Required sub-heading:
\subsubsection*{Qualifications and experience of the supervisor(s)}


\paragraph{Career development}
Employers and/or funders of researchers should draw up, preferably within the framework of their human resources management, a specific career development strategy for researchers at all stages of their career, regardless of their contractual situation, including for researchers on fixed-term contracts. It should include the availability of mentors involved in providing support and guidance for the personal and professional development of researchers, thus motivating them and contributing to reducing any insecurity in their professional future. All researchers should be made familiar with such provisions and arrangements.

\subsection{Capacity of the researcher to reach and re-enforce a position of professional maturity in research}
\label{sec:maturity}

