\documentclass[a4paper,11pt]{article}

%% Latex documents that need direct input
%% Stuff for MARIE CURIE
\usepackage[T1]{fontenc}
\usepackage{lmodern}
\usepackage{eurosym}
\usepackage{lastpage}
\usepackage{xspace}
\usepackage[margin=16mm,includehead,includefoot]{geometry}
\usepackage{fancyhdr}
\usepackage{booktabs}
\usepackage{multirow}
\usepackage{array}
\usepackage[table]{xcolor}
\usepackage{csquotes}
\usepackage{pgfgantt}
\usepackage{titlesec} % fancy titles for sections, subsections etc..
\usepackage{amssymb}
%% Regular stuff

%% Packages to help the writing/editing process
\usepackage{showkeys}
% \usepackage{showidx}
% todo notes
% This package has to be loaded after xcolor (take care that xcolor has to
% before some of the other included packages)
\usepackage{todonotes}

%  The following command loads a graphics package to include images
%  in the document. It may be necessary to specify a DVI driver option,
%  e.g., [dvips], but that may be inappropriate for some LaTeX
%  installations.
\usepackage[]{graphicx}

% In order to include files without having a clear page using \include*,
% the newclude package is required
\usepackage{newclude}

% Required for acronyms
% use \acresetall to reset the acroyms counter
% macros=True, allows for calling \myTriger rather than \ac{myTriger}
\usepackage[single=true, macros=true, xspace=true]{acro}

% Properly write units
% \usepackage{siunitx}

% Use biblatex to manage the referencing
%
\usepackage[style=verbose-ibid,backend=bibtex]{biblatex}
% \usepackage[style=authoryear,backend=biber]{biblatex}

% Clever cross referencing. Using cleverref, instead of writting
% figure~\ref{...} or equation~\ref{...}, only \cref{...} is required.
% The package interprates the references and introduces the figure, fig.,
% equation, eq., etc keywords. \Cref forces first letter capital.
% >> WARNING: This package needs to be loaded after hyperref, math packages,
%             etc. if used.
%             Cleveref is recomended to load late
\usepackage{hyperref}
\usepackage{cleveref}

% To create random text use lipsum
\usepackage{lipsum}
        % contains the latex packages
\input{content/frontmatter.tex}             % contains the information regarding Project call, title, autor, etc.
\input{latex/filesystem/fileSetup.tex}      % contains package and variables init.
%% Acronym definition example using glossaries package
%% \usepackage{acro} is required
%%
%% For a powerful usage of the acro package look at http://tex.stackexchange.com/questions/135975/how-to-define-an-acronym-by-using-other-acronym-and-print-the-abbreviations-toge
%
% \DeclareAcronym{a}{
%   short = ABC,
%   short-plural = AsBC,
%   long = Acronym Beautifuly Crafted,
%   long-plural = Acronyms Beautifuly Crafteds,
%   cite = {acro_man_url}
% }
%

\DeclareAcronym{mri}{
  short = MRI,
  long  = Magnetic Resonance Imaging
}

\DeclareAcronym{cemri}{
  short = CE-MRI,
  long  = Contrast-Enhanced \ac{mri}
}

\DeclareAcronym{dwi}{
  short = DWI,
  long  = Diffusion-Weighted Imaging
}

\DeclareAcronym{us}{
  short = US,
  long  = Ultra-Sound
}

\DeclareAcronym{nmle}{
  short = NMLE,
  long  = Non-mass-like enhancing
}

\DeclareAcronym{bsp}{
  short = BSP,
  long = Breast Screening Policy,
  long-plural = Breast Screening Policies
}

\DeclareAcronym{cad}{
  short = CAD,
  long  = Computer Aided Diagnosis
}

\DeclareAcronym{dm}{
  short = DM,
  long  = Digital Mammography
}

\DeclareAcronym{gt}{
  short = GT,
  long  = Ground Truth
}

\DeclareAcronym{bus}{
%  short = B\acs*{us},
%  long  = Breast \acifused{us}{\acs*{us}}{\acl*{us}}
short = BUS,
long= Breast Ultra-Sound
}

\DeclareAcronym{ml}{
  short = ML,
  long  = Machine Learning
}

\DeclareAcronym{svm}{
  short = SVM,
  long  = Support Vector Machines
}

\DeclareAcronym{acm}{
  short = ACM,
  long  = Active Contour Model
}

\DeclareAcronym{crf}{
  short = CRFs,
  long  = Conditional Random Fields
}

\DeclareAcronym{mrf}{
  short = MRFs,
  long  = Markov Random Fields
}

\DeclareAcronym{cv}{
  short = CV,
  long  = Computer Vision
}
\DeclareAcronym{icm}{
  short = ICM,
  long  = Iterated Conditional Modes
}
\DeclareAcronym{sa}{
  short = SA,
  long  = Simulate Anealing
}
\DeclareAcronym{gc}{
  short = GC,
  long  = Graph-Cuts
}

\DeclareAcronym{aov}{
  short = AOV,
  long  = Area Overlap
}

\DeclareAcronym{birads}{
  short = BI-RADS,
  long  = Breast Imaging-Reporting and Data System
}

\DeclareAcronym{mad}{
  short = MAD,
  long  = Median Absolute Deviation
}

\DeclareAcronym{qc}{
  short = QC,
  long  = Quadratic-Chi
}

\DeclareAcronym{sift}{
  short = SIFT,
  long  = Self-Invariant Feature Transform
}

\DeclareAcronym{bof}{
  short = BoF,
  long  = Back-of-Features
}

\DeclareAcronym{acr}{
  short = ACR,
  long  = American College of Radiology
}

\DeclareAcronym{fa}{
  short = FA,
  long  = Fibro-Adenoma
}

\DeclareAcronym{dci}{
  short = DCI,
  long  = Ductal Carcinoma in Situ
}

\DeclareAcronym{dic}{
  short = DIC,
  long  = Ductal Inflating Carcinoma
}

\DeclareAcronym{dcis}{
  short = DCIS,
  long  = Ductal Carcinoma in Situ
}

\DeclareAcronym{ilc}{
  short = ILC,
  long  = Inflating Lobular Carcinoma
}

\DeclareAcronym{fpr}{
  short = FPR,
  long  = False Positive Ratio
}

\DeclareAcronym{fnr}{
  short = FNR,
  long  = False Negative Ratio
}

\DeclareAcronym{fp}{
  short = FP,
  long  = False Positive
}

\DeclareAcronym{rbf}{
  short = RBF,
  long  = Radial Basis Function
}

\DeclareAcronym{vicorob}{
  short = ViCOROB,
  long = Computer Vision and ROBotics
}

\DeclareAcronym{florida}{
  short = FSU,
  long = Florida State University
}

\DeclareAcronym{udiat}{
  short = UDIAT,
  long = Centre Diagn\'{o}stic\,-\,Institud Universitary Parc Taul\'i\,-\,UAB
}

\DeclareAcronym{ivim}{
  short = IVIM,
  long  = Intravoxel Incoherent Motion
}

\DeclareAcronym{ecr}{
  short = ECR,
  long  = European Congress of Radiology
}

\DeclareAcronym{udg}{
  short = UdG,
  long  = Universitat de Girona
}

\DeclareAcronym{maia}{
  short = MaIA,
  long = MedicAl Imaging and Applications
  % http://maia-jointmaster.weebly.com/
}

\DeclareAcronym{ip}{
  short = IP,
  long  = Intellectual Property
}
      % contains the acronyms

%% Select inputing only one part of the document
%\includeonly{./content/participants/list}   % the file wihtout .tex
% \includeonly{./content/proposal/excelence}
\includeonly{./content/proposal/table}
%\includeonly{./content/proposal/impact}
%\includeonly{./content/proposal/implementation}
%\includeonly{./content/cv/cv}

\addbibresource{./content/lit_review.bib}

\begin{document}
\input{latex/first_page.tex}

\newpage
\setcounter{tocdepth}{1}
\setcounter{section}{-1}
\tableofcontents

\let\citeBk=\cite
\let\cite=\footcite

\section{LIST OF PARTICIPANTS}
\label{sec:participants}

\newcommand\rotx[1]{\rotatebox[origin=c]{90}{\textbf{#1}}}
\newcommand\roty[1]{\rotatebox[origin=c]{90}{\parbox{4cm}{\raggedright\textbf{#1}}}}
\newcommand\MyHead[2]{\multicolumn{1}{l|}{\parbox{#1}{\centering #2}}}

\renewcommand{\arraystretch}{2}
\noindent\begin{tabular}{|m{2.3cm}|m{1cm}|b{1em}|b{1em}|m{1.3cm}|m{2.5cm}|m{1.8cm}|m{4cm}|}
\rowcolor{lightgray}
\hline
  \textbf{Participants}
& \MyHead{1cm}{\cellcolor{lightgray}\textbf{Legal\\Entity\\Short\\Name}}
& \rotx{Academic}
& \rotx{Non-academic}
& \textbf{Country}
& \MyHead{2.1cm}{\cellcolor{lightgray}\textbf{Dept. / \\Division / \\Laboratory}}
& \textbf{Supervisor}
& \MyHead{2.5cm}{\textbf{Role of\\Partner\\Organisation}} \\
\hline
\underline{Beneficiary} & & & & & & & \\\hline
% Table Elements to fill use \checkmark where applicable.
%
% Full institution Name &
% Legal entity short name &
% Academic (check)&
% Non-Academic (check)&
% County&
% Dep./ Division/ Lab&
% Supervisor&
% Role of partner organization&
% \\\hline

Universitat de Girona &
UdG &
\checkmark &
 &
Spain &
Computer Vision and Robotics institute (ViCOROB) &
Dr.\,Joan Mart\'{i} &
\\\hline

\underline{Partner} \underline{Organisation} & & & & & & & \\\hline

Florida State University&
FSU&
\checkmark&
&
USA&
Scientific Computing&
Dr.\,Anke Meyer-Baese&
Host Outgoing phase
\\\hline

UDIAT - Centre Diagn\`{o}stic - Institud Universitari Parc Taul\'i - UAB&
UDIAT&
&
\checkmark&
Spain&
Dept. of breast and gynaecological radiology&
Dr. Melcior Sent\'is&
image acquisition, expert radiologist's feedback and clinical validation
\\\hline

\end{tabular}
\renewcommand{\arraystretch}{1}
\vspace{\baselineskip}

Data for non-academic beneficiaries

\noindent\begin{tabular}{|m{3cm}|m{2cm}|m{1.8cm}|c|c|c|c|c|c|}
\hline
  \textbf{Name}
& \roty{Location of research premises (city / country)}
& \roty{Type of R\&D activities}
& \roty{No. of fulltime employees}
& \roty{No. of employees in R\&D}
& \roty{Website}
& \roty{Annual turnover (approx. in Euro)}
& \roty{Enterprise status (Yes/No)}
& \roty{SME status  (Yes/No)}
\\\hline

UDIAT - Centre Diagn\`{o}stic - Institud Universitari Parc Taul\'i - UAB&
Sabadell, Spain&
Medical research&
350&
2&
\rotx{www.tauli.cat/udiat/}&
15 millions&
yes&
no
\\\hline
\end{tabular}
\vspace{\baselineskip}
                 % the file wihtout .tex

\newpage
\section{SUMMARY}
\label{sec:summary}

Breast cancer is the leading cause of cancer deaths among females worldwide.%~\cite{cancerStatistics2011}.
Nevertheless, death by breast cancer are highly reduced by early treatment.
Thus, to run a chance of surviving breast cancer, it is uttermost important the early detection of malignant tumors.
This has motivated the establishment of \acp{bsp} to facilitate this breast cancer detection at an early stage.
Despite X-ray \dm is considered the gold standard technique for \bsp, other screening techniques like \us and \mri are being investigated
to overcome \dm limitations due to tissue superposition which can either mimic or obscure malignant pathology,
and avoid X-ray radiation all together.

From the different \dm alternatives, the most promising to overcome the aforementioned limitations is \mri.
However, \nmle lesions exhibit a heterogeneous appearance in breast \mri with high variations in kinetic characteristics and typical morphological parameters, and resulting in a lower
reported specificity (69\%) and sensitivity (75\%) than mass-enhancing lesions.
Combinations of morphological and temporal \acs{birads} descriptors have proven to be insufficient to aid in the automated differential diagnosis of these lesions in \cemri.
Newest clinical studies suggest that T2-weighted image sequences and \dwi may provide additional specificity.

The aim of this fellowship is to translate these findings into a new \cad system.
%We propose to develop novel descriptors derived from the multispectral \mri data that will lead to a substantial improvement in diagnostic accuracy and efficiency and validate them in four specific experiments.
Our hypothesis is that a combination of novel descriptors extracted from multiparametric breast \mri has the potential to substantially improve the diagnostic value of the detection and classification of \acp{nmle}.

This first and novel \cad system in multiparametric breast \mri will reduce false positive recalls and thus increase specificity.
 A reduction in recall of only 5\% would already be clinically relevant, considering the costs and patient discomfort associated with second look ultrasound examinations and biopsies.

The experience of \acs{vicorob} in Breast-\cad, the preliminary studies in multispectral \mri carried out in \acs{florida} at the scientific computing division, and the clinical support from \acs{udiat} guarantee the success of this project as well as the correct transfer of knowledge from the laboratory to the clinical site.
It is also planned to commercialise the output \cad tool to clinical sites through existing medical imaging companies or via a spin-off.


The specific aims of this proposed project are to:
\begin{description}
\item [Aim\,1:]
  \hfill \\
Develop an image regularization framework for multiparametric breast \mri that includes a novel simultaneous elastic registration and segmentation algorithm.
  \emph{\textbf{Impact:} This methodology is fundamental for a correct image regularization and dramatically impacts the correct subsequent detection and diagnosis of \nmle lesions.}

  \item [Aim\,2:]
  \hfill \\
  Develop and apply novel image descriptors for characterizing lesion heterogeneity in T2-weighted \mri and \dwi.
  \emph{\textbf{Impact:} A combination of these image descriptors may increase the diagnostic value of existing \cad systems in breast \mri.}

  \item [Aim\,3:]
  \hfill \\
    Develop spatio-temporal feature extraction algorithms in \cemri.  \emph{\textbf{Impact:} These algorithms from Aim 2 and 3 will facilitate the categorization of \acp{nmle} lesions.}

  \item [Aim\,4:]
  \hfill \\
    Evaluation of the \cad system in terms of performance compared to trained readers and gold standard.
    \emph{\textbf{Impact:} Radiologists can benefit from this system by reduced interobserver variation and improved interpretation of breast \acp{mri} for the presence or absence of malignant non-mass-like enhancing lesions.}

  % \item [Aim\,5: ]
  % \hfill \\
  % \emph{\textbf{Impact:} }
\end{description}


%% Incldue the content without .tex extension
% \acresetall  % reset the acronyms from the abstract
\include*{./content/proposal/excelence}               % the file wihtout .tex
\include*{./content/proposal/impact}                  % the file wihtout .tex
\include*{./content/proposal/implementation}          % the file wihtout .tex
\include*{./content/proposal/table}          % the file wihtout .tex
\include{./content/cv/cv}                             % the file wihtout .tex
\include{./content/participants/capacities}           % the file wihtout .tex
\include{./content/ethical/info}                      % the file wihtout .tex

% \include*{content/other/other}
% \printbibliography

\input{latex/last_page.tex}

\newpage
\listoftodos
\end{document}
